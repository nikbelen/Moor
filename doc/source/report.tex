%%% Для сборки выполнить 2 раза команду: pdflatex <имя файла>

\documentclass[a4paper,12pt]{article}

\usepackage{ucs}
\usepackage[utf8x]{inputenc}
\usepackage[russian]{babel}
%\usepackage{cmlgc}
\usepackage{graphicx}
\usepackage{listings}
\usepackage{xcolor}
\usepackage{titlesec}
%\usepackage{courier}

\makeatletter
\renewcommand\@biblabel[1]{#1.}
\makeatother

\newcommand{\myrule}[1]{\rule{#1}{0.4pt}}
\newcommand{\sign}[2][~]{{\small\myrule{#2}\\[-0.7em]\makebox[#2]{\it #1}}}

% Поля
\usepackage[top=20mm, left=30mm, right=10mm, bottom=20mm, nohead]{geometry}
\usepackage{indentfirst}

% Межстрочный интервал
\renewcommand{\baselinestretch}{1.50}

% ------------------------------------------------------------------------------
% minted
% ------------------------------------------------------------------------------
\usepackage{minted}


% ------------------------------------------------------------------------------
% tcolorbox / tcblisting
% ------------------------------------------------------------------------------
\usepackage{xcolor}
\definecolor{codecolor}{HTML}{FFC300}

\usepackage{tcolorbox}
\tcbuselibrary{most,listingsutf8,minted}

\tcbset{tcbox width=auto,left=1mm,top=1mm,bottom=1mm,
right=1mm,boxsep=1mm,middle=1pt}

\newtcblisting{myr}[1]{colback=codecolor!5,colframe=codecolor!80!black,listing only, 
minted options={numbers=left, style=tcblatex,fontsize=\tiny,breaklines,autogobble,linenos,numbersep=3mm},
left=5mm,enhanced,
title=#1, fonttitle=\bfseries,
listing engine=minted,minted language=r}

%%%%%%%%%%%%%%%%%%%%%%%%%%%%%%%%%%%%%%%

\begin{document}

%%%%%%%%%%%%%%%%%%%%%%%%%%%%%%%
%%%                         %%%
%%% Начало титульного листа %%%

\thispagestyle{empty}
\begin{center}


\renewcommand{\baselinestretch}{1}
{\large
{\sc Петрозаводский государственный университет\\
Институт математики и информационных технологий\\
Кафедра информатики и математического обеспечения
}
}

\end{center}


\begin{center}
%%%%%%%%%%%%%%%%%%%%%%%%%
%
% Раскомментируйте (уберите знак процента в начале строки)
% для одной из строк типа направления  - бакалавриат/
% магистратура и для одной из
% строк Вашего направление подготовки
%
 Направление подготовки бакалавриата \\
% 01.03.02 --- Прикладная математика и информатика \\
% 09.03.02 --- Информационные системы и технологии \\
09.03.04 --- Программная инженерия \\
%%%%%%%%%%%%%%%%%%%%%%%%%
\end{center}

\vfill

\begin{center}
{\normalsize 
	Отчет по практике}

\medskip

%%% Название работы %%%
	{\Large \sc {Разработка мобильного приложения \\ <<Moor(101)>> }} \\
\end{center}

\medskip

\begin{flushright}
\parbox{11cm}{%
\renewcommand{\baselinestretch}{1.2}
\normalsize
	Выполнил:\\
% Выполнили:\\
%%% ФИО студента %%%
студент 2 курса группы 22207
\begin{flushright}
	Н. Д. Беленков \sign[подпись]{4cm}
\end{flushright}

%%% Второй участник %%%
% студента 1 курса группы 2210X
% \begin{flushright}
% 	И. О. Фамилия \sign[подпись]{4cm}
% \end{flushright}

%%%%%%%%%%%%%%%%%%%%%%%%%
% девушкам применять "Выполнила" и "студентка"
%%%%%%%%%%%%%%%%%%%%%%%%%
}
\end{flushright}

\vfill

\begin{center}
\large
    Петрозаводск --- 2022
\end{center}

%%% Конец титульного листа  %%%
%%%                         %%%
%%%%%%%%%%%%%%%%%%%%%%%%%%%%%%%

%%%%%%%%%%%%%%%%%%%%%%%%%%%%%%%%
%%%                          %%%
%%% Содержание               %%%

\newpage

\tableofcontents

%%% Содержание              %%%
%%%                         %%%
%%%%%%%%%%%%%%%%%%%%%%%%%%%%%%%

%%%%%%%%%%%%%%%%%%%%%%%%%%%%%%%%
%%%                          %%%
%%% Введение                 %%%

%%% В введении Вы должны описать предметную область, с которой связана %%%
%%% Ваша работа, показать её актуальность, вкратце определить цель     %%%
%%% разработки					       %%%


\newpage
\section*{Введение}
\addcontentsline{toc}{section}{Введение}

Цель проекта: Разработать мобильное приложение, дающее возможность играть в <<Мавр>> (также известа как <<101>>, <<Чешский дурак>> и т.д.).

Задачи проекта: 
\begin{enumerate} 
    \item Разработать прототип приложения с применением C++.
    \item Спроектировать интерфейс приложения.
    \item Перенести логику приложения на Kotlin.
    \item Реализовать интерфейс приложения на Kotlin.
    \item Проверить возжность использования приложения для игры.
\end{enumerate}

Мобильные игры являются наиболее крупным и перспективным рынком в мире гейминга. При разработке мобильных игр применяются специальные игровые движки, а также языки, используемые для разработки мобильных приложения - Java и Kotlin, последний язык появился не так давно(в 2011 году) и представляет для автора наибольший интерес среди представленных средств разработки.

Для того, чтобы получить опыт в мобильной разработке я решил реализовать на Kotlin несложную карточную игру <<Мавр>>.Таким образом, целью данного проекта стала разработка приложения, позволяющего проводить время за игрой в <<Мавр>>, ставший популярным в России еще в 19 веке. Данная карточная игра является весьма интересной и не забытой до сих пор.
\newpage
\section{Требования к приложению}
С точки зрения пользователя приложение должно иметь следующие свойства :
\begin{itemize}
    \item Возможность игры с соблюдением основных правил
    \item Удобный и понятный интерфейс
\end{itemize}

\newpage
\section{Проектирование приложения}

Вся логика приложения реализована в одном модуле:
\begin{itemize}
    \item MainActivity - включает в себя следующие функции:
    \begin{enumerate}
        \item addCardToPlayer() - добор карты из колоды игроком
        \item addCardToAI() - добор карты из колоды ИИ
        \item playCard() - ход выбранной картой
        \item tryToPlay() - попытка сходить выбранной картой (с последующими проверками)
        \item aiTurn() - логика ИИ
        \item canMove() - проверка возможности сходить данной картой
        \item setCardImage() - установка изображения для объекта
        \item endGame() - блокировка кнопок при окончании игры
    \end{enumerate}
\end{itemize}

\newpage
\section{Реализация приложения}
Для реализации приложения были использованы язык Kotlin, среда разработки Android Studio.

Немного статистики о реализации:
\begin{itemize}
    \item Количество модулей : 1
    \item Количество строк кода в MainActivity : 384
    \item Количество переменных : 16
    \item Количество функций : 8
    \item Количество файлов в ресурсах : 39
\end{itemize}

\newpage
\section*{Заключение}
\addcontentsline{toc}{section}{Заключение}
В результате нам удалось разработать приложение, которое позволяет играть в <<Мавр>>, также известный как <<101>>, игра следует основным правилам, описанным в приложении А, и имеет удобный интерфейс, адаптирующийся под различное количество карт на руке игрока. Итоговое приложение реализовано полностью с помощью Android Studio и языка Kotlin. 

Были использованы различные возможности Android Studio для реализации интерфейса : TextView, ImageView, ImageButton. ScrollView, различные варианты Layout.
\parВ ходе данной работы я получил опыт работы функциями языка Kotlin - основы программирования с помощью языка, в т.ч. ветвления, циклы, контейнеры и т.д., а также с возможностями Android Studio по созданию интерфейса мобильного приложения.

\newpage

\section*{Приложение А. Hello World}
\addcontentsline{toc}{section}{Приложение А. Правила игры в <<Мавр>>}
Игра имеет следующие основные правила: 
\begin{enumerate}
    \item Используется колода из 36 карт
    \item Сдаются карты по одной, 5 карт каждому игроку. Колода кладётся на середину стола. В первой игре первый ход достаётся случайно выбранному игроку, а далее игроку, победившему в прошлой игре. В некоторых вариантах первый ход совершает сдатчик выкладывая свою последнюю карту на стол.
    \item Некоторые карты требуют от игрока особых действий: \begin{itemize}
        \item Туз приводит к пропуску хода следующим игроком.
        \item Даму можно положить на любую карту, независимо от масти. Игрок положивший даму может заказать любую масть, а ход переходит к следующему игроку.
        \item Семёрка обязывает следующего игрока взять две карты из колоды и пропустить ход.
        \item Шестерка обязывает следующего игрока взять одну карту из колоды и пропустить ход.
        
    \end{itemize}
\end{enumerate}

\end{document}
